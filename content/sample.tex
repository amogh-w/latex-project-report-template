\chapter{Introduction}

\section{<section here>}

\subsection{<sub section here>}

\blindtext

\subsubsection{<sub sub section here>}

\blindtext

%some text\cite{citation-1-name-here}, some more text
%even more text\footnote{<footnote here>}, and even more.

\section{Adding Content}

\subsection{Images}

\begin{figure}[ht]
    \begin{center}
        \includegraphics[width=0.50\textwidth]{./vit-logo.png}\\[0.5cm]
        \caption{Sample Image.}
        \label{fig:sample-image}
    \end{center}
\end{figure}

\noindent To view the image code please refer to the `sample.tex' file.

\subsection{Tables}

\begin{table}[ht]
    \caption{Sample Table}
    \centering
    \begin{tabular}{ | c | c | c | }
        \hline
        cell1 & cell2 & cell3 \\
        \hline
        cell1 & cell5 & cell6 \\
        \hline
        cell7 & cell8 & cell9 \\
        \hline
    \end{tabular}
    \label{tab:sample-table}
\end{table}

\noindent To view the table code please refer to the `sample.tex' file.

\subsection{References and Citations}

You can refer to an image as Fig. \ref{fig:sample-image} and to a table as Table. \ref{tab:sample-table}.

\noindent You can cite items from the `references.bib' file as \cite{10.5120/ijca2020920393}.